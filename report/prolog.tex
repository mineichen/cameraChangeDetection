\chapter{Abstract}
Because recalibration of multi camera systems is computationally intensive, unnecessary recalculations should be avoided. In this project, an algorithm to detect camera movements was developed, which triggers a ChangeStart-Event when a camera starts to move and a ChangeStop-Event when the camera stabilizes in order to know when a revaluation is required. The difficulty lies in the constraint, that movements in the scene should not trigger a change.\newline
 
Because the software is supposed to run on a raspberry pi,  "Halide", a programming language dedicated for image processing was used to achieve the necessary performance. In order to integrate the software into existing code, the algorithm is implemented as a plugin for the gstreamer-1.0 framework, which is used by the research departement for computer vision at mid Sweden university. \newline

The resulted algorithm splits the image horizontally and vertically into a configurable amount of subsections. After filtering the entire image with a algorithm inspired by harris corner detection, the position of the pixel with the highest harris value in each subsection is stored in a vector. Subsequent images in the video-Stream just calculate the Harris corner value for these specific coordinates. If moore than 50\% of these coordinates are not considered to be corners anymore, the algorithm triggers a ChangeStart-Event. The changeStop-event is fired, when three subsequent images containt the same amount of edges. \newline

Analysis with different scenes demonstrate that no change of the camera was missed and less than 1\textperthousand \ of all frames erroneously triggered a change.\newline

Keywords: Raspicam, GSteamer, Halide, Camera-Change-Detection, Harris-corner-detection